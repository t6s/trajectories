\documentclass{easychair}

\usepackage{fullpage}

\title{Safe smooth paths between straight line obstacles}
\author{Yves Bertot}
\institute{Inria Université Côte d'Azur}
\begin{document}
\maketitle
We wish to present a comprehensive Coq program
to compute trajectories for a point between obstacles that are given by straight
line segments.  The Coq program contains four phases:
\begin{enumerate}
\item The obstacles are sorted in a sequence of {\em events}
\item The working space is decomposed into cells that are guaranteed to not contain
 any obstacle,
\item Given a source and a target point, a straight line trajectory is generated,
when possible
\item The angles of the straight line trajectory are then rounded using Bezier curves
\end{enumerate}
This algorithm works under the assumptions that the only intersections between segments
are at their extremities and there are no vertical obstacles.  preliminary reflexions
indicate that the first assumption can be handled by an extra algorithm, while
the second assumption can be handled by simple post processing of the available
connections between cells.

The algorithm receives as inputs the top and bottom edges, the sequence of segments,
and two points describing the source and target of the expected trajectory.  All
points used as obstacle extremities or as source and target for the trajectory are
given by cartesian coordinates.  All computations can be done using rational numbers,
but proofs require a more advanced number system (the field must at least
be real closed).

The result is either a sequence of trajectory elements or an exceptional value, called
{\tt None} , based
on the well-known {\tt option} datatype.  When the result is not {\tt None},
the following properties are guaranteed:
\begin{itemize}
\item Each of the trajectory elements is either a segment or Bezier curve,
\item The sequence of trajectory elements form a continuous path,
\item None of the trajectory elements has an intersection with either
the {\tt top} and {\tt bottom} edges or the obstracles.
\end{itemize}
This is work in progress and not all proofs have been completed.

\section{events}
The first and second phase combine to implement a {\em vertical cell decomposition}.
The mental model used to explain this algorithm is that a sweeping line moves from
left to right and stops every time an obstacle extremity is encountered.

To mimick this sweeping operation, we pre-process the segments
to obtain a sorted list of events, where each event is essentially the extremity
of one of the segments describing the obstacles.  This consists in taking as input
the list of obstacles and producing as output the sorted list of events.
\section{Vertical cells}
When processing the sequence of events, a sequence of open cells is maintained and a set
of closed cells is produced.  The open cells are cells where
the left-hand side is already known, but the right-hand side is not yet known because
the right extremities of the cells below and above these cells have not been process
yet.  The invariants that we expect for these collections are as follows:
\begin{itemize}
\item The open cells span the whole vertical line between the bottom and top edges
and they are vertically sorted
\item The high and low edges for each open cell encounter the sweeping line
\item All obstacles whose left point is left of the sweeping line are covered by the
top edge of one of the closed or open cells
\item All cells, whether open or closed are pairwise disjoint
\end{itemize}
The processing works by decomposing the sequencce of open cells into three sub-sequences
where the middle one contains all open cells that are in contact with the event being
processed.  These contact cells are closed and a new subsequences of cells is produced by
processing the list of obstacles whose left point is given by this event.  Special care
is taken to guarantee that the closed cells are non-empty, the problem happens when
two successive events are vertically aligned.
\section{Broken line trajectory}
Once the vertical cell decomposition is finished, we can start drawing trajectories
between arbitrary points in the work space.  This is done by finding the cells containing
the source and target points and then jumping from cell to neighboring
cell.  This is performed with a simple breadth-first search algorithm in the discrete graph
where the nodes are the cells and the edges connect neighbors.

We call {\em doors} the vertical boundaries between two
adjacent cells, and we can easily transform a discrete path from cell to cell into a discrete
path between doors.  Special care must be taken when entering and leaving a cell
through the same vertical.  In this case, we add a visit to the cell center.

The result is a sequence of straight line segments going from
door center to door center, with an occasional visit to a cell center.
The correctness of this trajectory is based on the property that it is
safe to move from a door into a cell to a door out of the cell.
\section{Making a smooth trajectory}
We use Bezier curves to smoothen each change of direction, and we verify that
the Bezier curves still remain in safe territory.

Each Bezier curve is defined by a sequence of control points, with the properties that 
the curve connects the first and last control point, it is contained in the convex hull
of the control points, and it is tangent to the segments
given by the first two and last two control points.  The first property helps guaranteeing
that the curves are trajectory still connects the same source and taget, the second property
is used to prove that the trajectory does collide with the obstacles.

In our program, we first produce a candidate Bezier curve for each corner, and
we then check whether this curve is safe.  If not, we produce Bezier curves,
that are closer to the broken line and repeatedly check their safety.  After a limited number
of iterations, the program falls back to the initial broken line corner if need be.
We only wish to prove that the checking procedure is correct.

Many of the elementary components were developed by collaborators: Thomas Portet
worked on the vertical cell decomposition algorithm and Quentin Vermande worked on
the general properties of Bezier curves and collisions with a convex hull.  Some
parts of the program also rely on finite maps provided by the Coq standard library.

\section{Exploiting the program and visual feedback}
This program can be run inside Coq, using rational numbers for the numbers system.
The output can be displayed by simply traversing the output data structure and generating
simple postscript commands for each of the straight line or Bezier curve segment.  Our
program uses quadratic Bezier curves, while postscript supports cubic Bezier curves, but
converting from one to the other is easy.  A Javascript user-interface for this geometric
program is also under study.
\end{document}


