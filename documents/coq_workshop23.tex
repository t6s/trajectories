\documentclass{easychair}

\usepackage{fullpage}

\begin{document}
\section{Introduction}
The goal of this presentation is to present a comprehensive Coq program
to compute trajectories for an object that has the size of a point between
obstacles that are given by straight line segments.  The Coq program
contains four phases:
\begin{enumerate}
\item The obstacles are sorted in a sequence of {\em events}
\item The working space is decomposed into cells that are guaranteed to not contain
 any obstacle,  through a process that consumes the events respecting the order
\item Given a source and a target point, astraight line trajectory is generated,
when possible, thanks to a breadth-first search algorithm
\item The angles of the straight line trajectory are then rounded using Bezier curves
\end{enumerate}
In its current form, the algorithme works under the following pre-condition:
\begin{itemize}
\item There are two segments {\tt top} and {\tt bottom}, which define a box containing
all the obstacles, with no contacts between the obstacles and these segments
\item The only intersections between segments are at their extremities
\item None of the obstacles is vertical (in other words, all segments link two points
where the first coordinate of one point is less than the first coordinate of the other)
\end{itemize}

The algorithm receives as inputs the top and bottom edges, the sequence of segments,
and two points describing the source and target of the expected trajectory.

The result is either a sequence of trajectory elements or an exceptional value, called
{\tt None} , based
on the well-known {\tt option} datatype.  When the result is not {\tt None},
the following properties are guaranteed:
\begin{itemize}
\item Each of the trajectory elements is either a segment or Bezier curve,
\item The sequence of trajectory elements form a continuous path,
\item None of the trajectory elements has an intersection with either
the {\tt top} and {\tt bottom} edges or the obstracles.
\end{itemize}
This is work in progress, so that not all of these properties have been proved yet.
We shall provide animations of the various phases of this program.

\end{document}
