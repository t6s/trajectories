\documentclass[compress]{beamer}
\usepackage[latin1]{inputenc}
\usepackage{alltt}

\setbeamertemplate{footline}[frame number]
\title{Smooth trajectories in straight line mazes}
\author{Yves Bertot}
\date{April 2023}
\mode<presentation>
\begin{document}
\maketitle
\begin{frame}
\frametitle{The game}
\begin{itemize}
\item Find a smooth path in a maze
\item Decompose the problem
\begin{itemize}
\item Find a discrete approximation of the problem
\item Construct a broken line (non-smooth path)
\item smoothen the angles
\end{itemize}
\item Prove the correctness of the algorithm
\begin{itemize}
\item Prove the absence of collision
\item work in progress
\item Ideally one should also prove that a path
is found as soon as one exists
\end{itemize}
\end{itemize}
\end{frame}
\begin{frame}
\frametitle{Example}
\includegraphics[trim={0 0 0 10cm}, clip, width=\textwidth]{empty_spiral.pdf}
\end{frame}
\begin{frame}
\frametitle{Cell decomposition}
\begin{itemize}
\item Decompose the space into simple cells
\item Each cell is convex
\item Each cell is free of obstacles
\item Each cell may have neighbours where moving is safe
\end{itemize}
\end{frame}
\begin{frame}
\frametitle{Vertical cell decomposition}
\begin{itemize}
\item Use a vertical sweep line moving left to right
\item Stop each time one meets an edge tip (an event)
\item maintain a vertically ordered sequence of open cells
\begin{itemize}
\item close all open cells in contact with the event
\item open new cells forall edges starting at this event
\end{itemize}
\item Simplifying assumptions
\begin{itemize}
\item No vertical edges
\item Edges do not cross
\end{itemize}
\end{itemize}
\end{frame}
\begin{frame}
\frametitle{Difficulty with vertically aligned events}
\begin{itemize}
\item  Closed cells may be degenerate
\begin{itemize}
\item  Left and right side are in contact
\end{itemize}
\item Solution: special treatment
\begin{itemize}
\item Add points to the right side of last closed cell
\item Add points to the left side of last opened cell
\end{itemize}
\end{itemize}
\end{frame}
\begin{frame}
\frametitle{Vertical cell decomposition example}
\includegraphics[trim={0 0 0 10cm}, clip, width=\textwidth]{cells_spiral.pdf}
\end{frame}
\begin{frame}
\frametitle{Cell assumptions}
\begin{itemize}
\item Vertical edges are safe passages between two cells
\item Moving directly left-edge right-edge is safe
\begin{itemize}
\item and vice-versa
\end{itemize}
\item Moving from a left-edge to the cell center is safe
\begin{itemize}
\item similarly for a right-edge
\item moving from left-edge to left-edge is safe by going through the
  cell center
\end{itemize}
\end{itemize}
\end{frame}
\begin{frame}
\frametitle{Finding a path in the cell graph}
\begin{itemize}
\item A discrete path from cell to cell is found by breadth-first search
\item Connected components of the graph are defined by polygons
\item Special care for points that are already on the common edge of two cells
\end{itemize}
\end{frame}
\begin{frame}
\frametitle{Two examples of elementary safe paths}
\includegraphics[trim={0 0 0 10cm}, clip, width=\textwidth]{spiral_safe2.pdf}
\end{frame}
\begin{frame}
\frametitle{Making a broken line path between points}
\begin{itemize}
\item Find the cells containg the points
\item Find a discrete path between cell names
\item Make a path from vertical edge midpoint to vertical edge midpoint
\item Connect the source and target point to the first and last
   vertical edge midpoints
\begin{itemize}
\item Unless the source or targets are themselves on a vertical edge
\end{itemize}
\end{itemize}
\end{frame}
\begin{frame}
\frametitle{broken line safe path between points}
\includegraphics[trim={0 0 0 10cm}, clip, width=\textwidth]{spiral_bline.pdf}
\end{frame}
\begin{frame}
\frametitle{Making corners smooth}
\begin{itemize}
\item Angles would require a robot to stop to turn
\item rounded bends makes it possible to keep moving
\item First approximation: no limit on steering radius
\item Using quadratic Bezier curves for this purpose
\end{itemize}
\end{frame}
\begin{frame}
\frametitle{The basics of quadratic B�zier curves}
\begin{itemize}
\item Bezier curves are given by a set of control points
   (3 for a quadratic curve)
\item Points on the curves are obtained by computing weighted barycenters
\begin{itemize}
\item The curve is enclosed in the convex hull of the control points
\end{itemize}
\item Given control points \(a_0, a_1, \ldots, a_{n-1}, a_n\), \(a_0, a_1\)
is tangent to the curve in \(a_0\)
\begin{itemize}
\item same for \(a_{n-1}, a_n\) in \(a_n\)
\end{itemize}
\end{itemize}
\end{frame}
\begin{frame}
\frametitle{Bezier curve illustration}
\begin{itemize}
\item Straight edge tips of this drawing are control points
\item The curve is inside the triangle
\end{itemize}
\includegraphics[trim={0 6cm 0 19cm}, clip, width=\textwidth]{bezier_example.pdf}
\end{frame}
\begin{frame}
\frametitle{Plotting the Bezier curve}
\begin{itemize}
\item Show how the point for ratio \(4/9\) is computed
\item Control points for the two subcurves are given by the new point,
the initial starting and end points, and the solid green straight edge tip
\end{itemize}
\includegraphics[trim={0 6cm 0 19cm}, clip, width=\textwidth]{bezier_example2.pdf}
\end{frame}
\begin{frame}
\frametitle{Using Bezier curves for smoothing}
\begin{itemize}
\item Add extra points in the middle of each straight line segment
\item Uses these extra points as first and last control points for Bezier curves
\item Use the angle point as the middle control point
\item Check the Bezier curve for collision and repair if need be
\end{itemize}
\end{frame}
\begin{frame}
\frametitle{Checking for collision}
\begin{itemize}
\item Two kinds of angles
\item Angles at cell center: in the middle of safe space
\begin{itemize}
\item No risk of collision
\end{itemize}
\item angles at vertical edge midpoint
\begin{itemize}
\item Use dichotomy until a guaranteed result is obtained
\item To compute control points in dichotomy, only half sums are needed
\end{itemize}
\end{itemize}
\end{frame}
\begin{frame}
\frametitle{Collision checking, graphically}
\includegraphics[trim={0 4cm 0 17cm}, clip, width=\textwidth]{collision.pdf}
\end{frame}
\begin{frame}
\frametitle{Not passing in the safe zone}
\includegraphics[trim={0 4cm 0 17cm}, clip, width=\textwidth]{collision2.pdf}
\end{frame}
\begin{frame}
\frametitle{Repairing a faulty curve}
\begin{itemize}
\item Simple solution: bring the control points closer together
\item Use the first half points computed in the checking phase
\item Try again with this curve
\end{itemize}
\end{frame}
\begin{frame}
\frametitle{Constructing a repaired curve}
\includegraphics[trim={0 4cm 0 17cm}, clip, width=\textwidth]{repair.pdf}
\end{frame}
\begin{frame}
\frametitle{Checking the repaired curve}
\includegraphics[trim={0 4cm 0 17cm}, clip, width=\textwidth]{repair2.pdf}
\end{frame}

\end{document}


%%% Local Variables: 
%%% mode: latex
%%% TeX-master: t
%%% End: 
